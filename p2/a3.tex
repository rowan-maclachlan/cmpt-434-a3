\documentclass[10pt]{amsart}
\usepackage{hyperref}
\usepackage{parskip,fullpage}
\usepackage{graphicx}
\usepackage{amsmath}
\usepackage{amsthm}
\usepackage[T1]{fontenc}
\usepackage{geometry}
\usepackage{siunitx}
\usepackage{multirow} % These packages are used for rendering adjacency matrices
\usepackage{rotating} 
\sisetup{output-exponent-marker=\ensuremath{\mathrm{e}}}

\geometry{
	body={7in, 9.5in},
	left=.5in,
	top=.50in
}

\newtheorem*{theorem}{Theorem}
\newtheorem{lemma}{Lemma}
\calclayout
\begin{document} \textbf{CMPT 434 Assignment 2} \\ Winter 2019\\
Due Date: Monday January 11th\\
rdm659 11165820

Part B

\begin{enumerate}
    \item (2 marks) Give the checksum for 10111010101 using CRC with
        generator 11011.\\
        $m = 10111010101$, $r = 4$, $G = 11011$\\
        We want to add $r$ (4) bits to $m$ such that $0 = (m / G)$
        Then, we need to calculate the remainder from dividing $G$ into $m + r$,
        and add that remainder to the transmitted frame before sending it.\\
        \begin{enumerate}
            \item Add $r$ zeroes to $m$\\
                $m+r \rightarrow 101110101010000$
            \item Divide $G$ into $m+r$\\
                $11011 / 101110101010000$ gives remainder 1011:\\
                \begin{enumerate}
                    \item $10111$ XOR $11011 \rightarrow 1100$
                    \item $11000$ XOR $11011 \rightarrow 11$
                    \item $11101$ XOR $11011 \rightarrow 110$
                    \item $11001$ XOR $11011 \rightarrow 10$
                    \item $10000$ XOR $11011 \rightarrow 1011$
                    \item $10110$ XOR $11011 \rightarrow 1101$
                \end{enumerate}
            \item 
                Then, subtract the remainder (1101) from $101110101010000$ by
                XOR to get $10111010101101$.
        \end{enumerate}
        Then, the transmitted frame is the bit string 10111010101101 and the
        checksum is 1101.
    \item (4 marks) Give the ratio of the propagation delay $\tau$ to the frame
        transmission time $t$ for each of the following wireless (i.e., use the
        speed of signal propagation through air) networking scenarios (round
        each answer off to 2 significant figures), and state for each which of
        ALOHA or CSMA would likely be more appropriate: 
    \begin{itemize}
        \item $\approx$ 30 meter distance between nodes, 100 Mbps data rate, 1
            Kbit frames.\\
            So if the signal travels at the speed of light to a receiver 30m
            away then $tau$ is $30m / \num{3.0E8}m/s = \num{1.0E-7}s$\\
            The frame transmission time is length of time from the the first bit
            to getting the last bit of a frame onto the network substrate - or,
            equivalently, the length of time from the first bit until we receive
            the last bit of a frame.\\
            In our case, a $1Kb$ over a $100 Mbps$ data rate gives a frame
            transmission time of $t = \num{1.0E-3}s$\\
            Then, the ratio of $tau/t = \num{1.0E-7}s / \num{1.0E-3}s =
            \num{1.0E-4}s$\\
            
            This is a small ratio and the value of $tau$ is quite small.  The
            frame transmission is not so small relative to the propogation
            delay.  This means that a receiver may be busy for a significant
            time relative to the time it takes to transmit a message.  Because
            the propogation delays are small, we could listen in to the activity
            on our channel to see when we could send - this implies that CSMA
            may acheive higher efficiencies than ALOHA and should be used in its
            place.
        \item $\approx$ 300 meter distance between nodes, 10 Gbps data rate, 4
            Kbit frames.\\
            $\tau$ is $300m / \num{3.0E8}m/s = \num{1.0E-6}s$\\ 
            The frame transmission time $t$ is $4Kb$ over a $10Gbps$ data rate
            which gives $t = 4Kb / 10Gbps = \num{4.0E-4}s$\\
            This ratio is $\tau/t = \num{1.0E-6}s / \num{4.0E-4}s =
            \num{2.5E-2}$\\
            This relatively larger ratio means that senders can do little to
            distinguish when the channel is really busy or not because of the
            large propogation delay.  The decrease transmission time means that
            receivers are not so busy comparatively.  Here, ALOHA may acheive
            higher efficiences than CSMA.
    \end{itemize}

    \item Consider the following graph in which the nodes represent routers, the
        edges represent links, and the labels indicate the link costs.
    \begin{enumerate}
        \item (2 marks) Show the shortest path length estimates after each step
            when Dijkstra’s algorithm is applied to the above network with node
            e as the source.\\
            \begin{tabular}{|lr|l|l|l|l|l|l|l|l|l|} \cline{3-9}
            \multicolumn{1}{l}{} && \multicolumn{7}{c|}{steps} \\ \cline{3-9}
            \multicolumn{1}{l}{} & & e & d & g & h & f & b & c \\ \hline
            \multirow{8}{*}{\begin{sideways}distance to 'e'\end{sideways}}
            %                           e   d   g   h   f   b   c  
            & \multicolumn{1}{|r|}{a} & 0 & 0 & 0 & 0 & 0 & 7 & 6 \\ \cline{2-9}
            & \multicolumn{1}{|r|}{b} & 8 & 4 & 4 & 4 & 4 & - & - \\ \cline{2-9}
            & \multicolumn{1}{|r|}{c} & 0 & 0 & 8 & 8 & 8 & 5 & - \\ \cline{2-9}
            & \multicolumn{1}{|r|}{d} & 2 & - & - & - & - & - & - \\ \cline{2-9}
            & \multicolumn{1}{|r|}{e} & - & - & - & - & - & - & - \\ \cline{2-9}
            & \multicolumn{1}{|r|}{f} & 6 & 5 & 5 & 4 & - & - & - \\ \cline{2-9}
            & \multicolumn{1}{|r|}{g} & 2 & - & - & - & - & - & - \\ \cline{2-9}
            & \multicolumn{1}{|r|}{h} & 0 & 0 & 3 & - & - & - & - \\ \hline
            \end{tabular}

        \item (2 marks) Suppose now that distance vector routing is used, and
            that each minute all nodes simultaneously send routing updates to
            their neighbours. Consider those updates relating to the path
            lengths to node $e$. Let $Dx(e)$ denote node $x$'s estimate of its
            shortest path length to node $e$. Each minute, each node $x$ sends
            $Dx(e)$ to each of its neighbours, and simultaneously receives
            $Dy(e)$ from each of its neighbour nodes $y$. Node $x$ will then
            update $Dx(e)$ to the minimum over all neighbours $y$ of $Dy(e)$
            plus the cost of the link between $x$ and $y$. Supposing that link
            $e$-$g$ fails, give a table showing the evolution of the $Dx(e)$
            values for all nodes $x$ (except $e$) until these values stabilize.
            For the first row of the table, use the shortest path lengths when
            there are no link failures. Note that because of the assumed
            synchronous operation, the $Dx(e)$ values in one row will reflect
            the current link costs together with the $Dy(e)$ values in the
            previous row.\\
            \begin{tabular}{|lr|l|l|l|l|l|l|l|l|l|} \cline{3-7}
            \multicolumn{1}{l}{} && \multicolumn{5}{c|}{steps} \\ \cline{3-7}
                \multicolumn{1}{l}{} & & 1 & 2 & 3 & 4 & 5 \\ \hline
            \multirow{8}{*}{\begin{sideways}distance to 'e'\end{sideways}}
                & \multicolumn{1}{|r|}{Da(e)} & 6 & - & - & - & - \\ \cline{2-7}
                & \multicolumn{1}{|r|}{Db(e)} & 4 & - & - & - & - \\ \cline{2-7}
                & \multicolumn{1}{|r|}{Dc(e)} & 5 & - & - & - & - \\ \cline{2-7}
                & \multicolumn{1}{|r|}{Dd(e)} & 2 & - & - & - & - \\ \cline{2-7}
                & \multicolumn{1}{|r|}{De(e)} & - & - & - & - & - \\ \cline{2-7}
                & \multicolumn{1}{|r|}{Df(e)} & 4 & - & - & 5 & 5 \\ \cline{2-7}
                & \multicolumn{1}{|r|}{Dg(e)} & 2 & 3 & 3 & 5 & 7 \\ \cline{2-7}
                & \multicolumn{1}{|r|}{Dh(e)} & 3 & 3 & 4 & 4 & 6 \\ \hline
            \end{tabular}

    \end{enumerate}
\end{enumerate}


\end{document}
