\documentclass[10pt]{amsart}
\usepackage{hyperref}
\usepackage{parskip,fullpage}
\usepackage{graphicx}
\usepackage{amsmath}
\usepackage{amsthm}
\usepackage[T1]{fontenc}
\usepackage{geometry}
\usepackage{multirow} % These packages are used for rendering adjacency matrices
\usepackage{rotating} 
\usepackage{listings}
\usepackage[detect-weight=true, binary-units=true]{siunitx}
\sisetup{output-exponent-marker=\ensuremath{\mathrm{e}}}

\geometry{
	body={7in, 9.5in},
	left=.5in,
	top=.50in
}

\newtheorem*{theorem}{Theorem}
\newtheorem{lemma}{Lemma}
\calclayout
\begin{document} \textbf{CMPT 434 Assignment 3} \\ Winter 2019\\
Due Date: Monday March 18th\\
rdm659 11165820

Part B

\begin{enumerate}
    \item (2 marks) Give the parameters of a token bucket that limits the long
        term average rate to 40 Mbps (i.e., $\num{40E6}$ bits per second) and the
        maximum duration of back-to-back packet transmissions to 500
        microseconds, assuming a link of capacity of 1 Gbps.

        Parameters of a token bucket are $B$, the capacity of the bucket, and
        $R$, the rate at which new tokens are 'added' or 'dispensed'.

        If we want to limit the long term average rate to 40 Mbps, then we
        should set $R = 40 Mbps$ to prevent the sender from ejecting more than
        40 Mb a second.

        If we want to limit the maximum duration of back to back packet
        transmission to 500 Microseconds, then we want our bucket to run out of
        tokens after 500 microseconds of consuming them at 1 Gbps.  We can use
        the following formula to calculate $B$:
        \[ B + RS = MS \]
        where $M$ is $\num{1E9}\si{\bit\per\second}$, $R$ is
        $\num{40E6}\si{\bit\per\second}$, and $S$ is
        $\num{5.0E-4}\si{\second}$.  To solve for $B$, the capacity of the bucket, we can re-write our
        formula as $B = MS - RS \iff B = S(M-R)$.  This gives:\\
         \[ B = \num{5.0E-4}\si{\second} (\num{1.0E9} \si{\bit\per\second} -
         \num{4.0E7} \si{\bit\per\second}) \]
         \[ B = \num{5.0E-4}\si{\second} * \num{9.6 E8} \si{\bit\per\second} \]
         \[ B = \num{4.8E5}\si{\bit} \]
        Here, we see that a token bucket with parameter of $R = 40
        \si{\mega\bit\per\second}$ and $B = 480000\si{\bit}$ achieves this.

    \item (8 marks) For each of the following destination IP addresses, state whether an
        incoming IP datagram would be forwarded on an outgoing interface using a
        destination link layer address belonging to a next hop router, or a link
        layer address belonging to the IP datagram’s destination host. In the
        former case, state which next hop router. Make sure you justify your
        answers.

        \begin{enumerate}
            \item 10.11.43.127\\
            \item 10.11.43.160\\
            \item 10.11.49.123\\
            \item 10.11.44.222\\
        \end{enumerate} 
    \item (2 marks) A TCP sender’s value of SRTT is 125 milliseconds, but then a
        routing change occurs, after which all measured RTTs are 80
        milliseconds. How many measurements of the new RTT are required before
        SRTT drops below 100 milliseconds? Assume that a weight of 0.125 is used
        for the new sample, and a weight of 0.875 for the old value, when
        updating SRTT. Make sure to show how you got your answer.

        The formula for calculating the SRTT based on rount-trip times is:
        \[ SRTT = \alpha SRTT + (1-\alpha) * R \]

        If we were to script a solution to this...

        \begin{lstlisting}[language=python]
            ALPHA = 7/8
            INITIAL_SRTT = 125
            
            def new_srtt(old_srtt, alpha, r):
                return alpha*old_srtt + (1-alpha)*r
            
            i = 0
            srtt = INITIAL_SRTT
            while(srtt > 100):
                print("Iteration " + str(i) + ": " + str(srtt))
                srtt = new_srtt(srtt, ALPHA, 80)
                i += 1
            
            print("Iteration " + str(i) + ": " + str(srtt))
        \end{lstlisting}

        ... and then run it, we would get the following output, showing that it
        would take 8 iterations to bring the value of SRTT below 80:

        Iteration 0: 125\\
        Iteration 1: 119.375\\
        Iteration 2: 114.453125\\
        Iteration 3: 110.146484375\\
        Iteration 4: 106.378173828125\\
        Iteration 5: 103.08090209960938\\
        Iteration 6: 100.1957893371582\\
        Iteration 7: 97.67131567001343\\

    \item (2 marks) Suppose that a TCP connection’s RTT is 125 milliseconds
        except for every N’th RTT, which is 500 milliseconds. Note that if N is
        sufficiently large, the 500 millisecond RTT will cause a timeout, since
        that RTT value will be highly “unexpected” (assume here that RTOmin is
        less than 500 milliseconds), while if N is sufficiently small, it won’t
        cause a timeout. What is the largest N for which the 500 millisecond RTT
        won’t cause a timeout? (Consider only the behavior after there have been
        many measured RTTs following this pattern, and the initial values chosen
        for SRTT and RTTVAR no longer have any significant impact.) Assume that
        the same weights are used as in question 3 when updating SRTT, and
        weights of 0.25 (new sample) and 0.75 (old value) when updating RTTVAR.
        Make sure to show how you got your answer.

    \item (14 marks) Suppose that a sender using TCP Reno is observed to have
        the following congestion window sizes, as measured in segments, during
        each transmission “round” spent in slow start or additive increase
        mode...

        \begin{enumerate}
            \item Identify the transmission rounds when TCP is in slow start
                mode.\\
                rounds 1-7, rounds 22-26
            \item After the 13th transmission round, is segment loss detected by
                a triple duplicate ACK or by a timeout?\\
                By a triple duplicate ACK, as the \textit{cwnd} was cut only in
                half.
            \item After the 20th transmission round, is segment loss detected by
                a triple duplicate ACK or by a timeout?\\
                After round 21, segment loss is detected by a timeout.  This can
                be determined by observing that \textit{cwnd} has fallen to 1 in
                round 22.
            \item What is the initial value of the slow start threshold ssthresh
                at the first transmission round?\\
                This value must be 64, as that is the value of \textit{cwnd}
                after which additive increase begins at round 8.
            \item What is the value of ssthresh at the 18th transmission
                round?\\
                Assuming \textit{ssthresh} is updated to be equal to the last
                highest value of \textit{cwnd} acheived before packet loss,
                \textit{ssthresh} must be 70.
            \item What is the value of ssthresh at the 24th transmission
                round?\\
                \textit{ssthresh} was likely updated to a value of 42 after
                packet loss was detected in round 21 with \textit{cwnd} equal to
                42.
            \item Assuming that a segment loss is detected after the 26th round
                by the receipt of a triple duplicate ACK, what will be the new
                value of ssthresh and cwnd?\\
                \textit{ssthresh} will be set to 16, the last highest value of
                \textit{cwnd} before packet loss, and \textit{cwnd} will be set
                to 8, half the previous value of \textit{cwnd}.
        \end{enumerate}
    \end{enumerate}
\end{document}
